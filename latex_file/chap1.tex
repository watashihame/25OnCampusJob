\section{概要}
Dockerは,オープンソースのアプリケーションコンテナエンジンであり,Go言語で実装され,Apache 2.0ライセンスのもとで公開されている.Dockerを使用すると,開発者が作成したアプリケーションとその実行環境を簡単に移行できる.
実験において,多くの場合,異なる論文で提案された手法を検証する必要があるが,これらの手法は環境に大きく依存することがよくある.最近では,多くの研究者が標準的な環境構築手順に加え,Dockerコンテナを提供している.
直接PCやサーバー上で依存関係を構築すると,以下のような問題がある:
\begin{itemize}
    \item 既存のソフトウェア環境(例:ハードウェアドライバのバージョンやコンパイラのバージョン)に影響を与える.
    \item 手動での環境構築には時間と労力がかかる(特に,一部のソフトウェアはソースコードからのコンパイルが必要).
\end{itemize}

仮想マシン(VM)イメージを使用する場合,構築に時間がかかり,性能のオーバーヘッドがある.しかし,Dockerは次のような利点を持っている:
\begin{itemize}
    \item ホストマシン上に「仮想環境」を構築し,既存のソフトウェア環境に影響を与えない.
    \item 依存関係を簡単に構築できるため,手間がかからず,アプリケーションの展開が容易.
    \item 性能オーバーヘッドが非常に低く,ほぼ全ての計算リソースを使用可能.
\end{itemize}

このチュートリアルでは,Ubuntu 24.04 LTSを基にしている.読者が基本的な\texttt{Linux}コマンドに関する知識を持っていると,本文はより理解しやすくなる.このチュートリアルには次の内容が含まれている:
\begin{enumerate}
    \item Docker Engineのインストール
    \item Dockerを使用した深層学習アプリケーションのデプロイ
    \item \texttt{docker-compose}を使用したアプリケーションのデプロイ
    \item Dockerコンテナを用いた開発環境の構築
\end{enumerate}
\section{Docker Engineのインストール}
Dockerはバージョン17.03以降,コミュニティ版(CE:Community Edition)とエンタープライズ版(EE:Enterprise Edition)を提供している.本チュートリアルでは,CE版を使用する.
\subsection{公式スクリプトによるインストール}
\begin{lstlisting}[language=bash]
curl -fsSL https://get.docker.com -o get-docker.sh
sudo sh ./get-docker.sh --dry-run
\end{lstlisting}
\paragraph{注意:}
\begin{itemize}
    \item \texttt{root} 権限が必要.
    \item スクリプトは自動的にシステムのバージョンを認識し,パッケージ管理を設定する.
    \item インストール時のパラメータ設定は不可だが,本チュートリアルでは特に問題ない.
    \item スクリプトはDockerの最新バージョンをインストールするが,Dockerのアップグレードには使用できない.
\end{itemize}
\subsection{\texttt{apt}を使用したインストール}
\begin{enumerate}
    \item Dockerのリポジトリを設定
\begin{lstlisting}[language=bash]
# Add Docker's official GPG key:
sudo apt-get update
sudo apt-get install ca-certificates curl
sudo install -m 0755 -d /etc/apt/keyrings
sudo curl -fsSL https://download.docker.com/linux/ubuntu/gpg -o /etc/apt/keyrings/docker.asc
sudo chmod a+r /etc/apt/keyrings/docker.asc
# Add the repository to Apt sources:
echo \
  "deb [arch=$(dpkg --print-architecture) signed-by=/etc/apt/keyrings/docker.asc] https://download.docker.com/linux/ubuntu \
  (. /etc/os-release && echo \"(. /etc/os-release && echo \"{UBUNTU_CODENAME:-$VERSION_CODENAME}\") stable" | \
  sudo tee /etc/apt/sources.list.d/docker.list > /dev/null
sudo apt-get update
\end{lstlisting}
    \item 最新バージョンのDockerをインストール
\begin{lstlisting}[language=bash]
sudo apt-get install docker-ce docker-ce-cli containerd.io docker-buildx-plugin docker-compose-plugin
\end{lstlisting}
\end{enumerate}

アップデート時は,手順2を繰り返すだけでOK.
\subsection{インストール確認}
\begin{lstlisting}[language=bash]
sudo docker run hello-world
\end{lstlisting}

出力結果は以下のようになる:
\begin{lstlisting}
Hello from Docker!
This message shows that your installation appears to be working correctly.
\end{lstlisting}
\subsection{ユーザーのDocker実行権限を設定}
上記の手順でインストールされたDockerは,デフォルトでは\texttt{root}ユーザーとして実行される.そのため,\texttt{sudo}を毎回使用する必要がある.

Dockerをより便利に使用するために,現在のユーザーを\texttt{docker}グループに追加する.このグループはDockerのインストール時に自動的に作成される.
\begin{enumerate}
    \item \texttt{docker}グループの確認と作成
\begin{lstlisting}[language=bash]
getent group docker
\end{lstlisting}
もしグループが存在しない場合,以下のコマンドで作成する:
\begin{lstlisting}[language=bash]
sudo groupadd docker
\end{lstlisting}
    \item ユーザーを\texttt{docker}グループに追加する:
\begin{lstlisting}[language=bash]
sudo usermod -aG docker your_username
newgrp docker
\end{lstlisting}
\end{enumerate}

これにより,今後\texttt{sudo}を使わずにDockerコマンドを実行できる.