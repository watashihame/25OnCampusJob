
\section{概要}
Dockerは、オープンソースのアプリケーションコンテナエンジンであり、Go言語で実装され、Apache 2.0ライセンスのもとで公開されています。Dockerを使用すると、開発者が作成したアプリケーションとその実行環境を簡単に移植できます。

実験において、多くの場合、異なる論文で提案された手法を検証する必要がありますが、これらの手法は環境に大きく依存することがよくあります。最近では、多くの研究者が標準的な環境構築手順に加え、Dockerコンテナを提供するようになっています。

直接PCやサーバー上で依存関係を構築すると、以下のような問題が発生します:
\begin{itemize}
    \item 既存のソフトウェア環境(例:ハードウェアドライバのバージョンやコンパイラのバージョン)に影響を与える。
    \item 手動での環境構築には時間と労力がかかる(特に、一部のソフトウェアはソースコードからのコンパイルが必要)。
\end{itemize}

仮想マシン(VM)イメージを使用する場合、構築に時間がかかり、性能のオーバーヘッドが発生します。しかし、Dockerは次のような利点を提供します:
\begin{itemize}
    \item ホストマシン上に「仮想環境」を構築し、既存のソフトウェア環境に影響を与えない。
    \item 依存関係を簡単に構築できるため、手間がかからず、アプリケーションの展開が容易。
    \item 性能オーバーヘッドが非常に低く、ほぼ全ての計算リソースを使用可能。
\end{itemize}

このチュートリアルでは、Ubuntu 24.04 LTSを前提とします。読者は基本的な\texttt{Linux}コマンドに関する知識を持っていることが求められます。

\section{Docker Engineのインストール}
Dockerはバージョン17.03以降、コミュニティ版(CE:Community Edition)とエンタープライズ版(EE:Enterprise Edition)を提供しています。本チュートリアルでは、CE版を使用します。

\subsection{公式スクリプトによるインストール}
\begin{lstlisting}[language=bash]
curl -fsSL https://get.docker.com -o get-docker.sh
sudo sh ./get-docker.sh --dry-run
\end{lstlisting}
\textbf{注意点}:
\begin{itemize}
    \item \texttt{root} 権限が必要です。
    \item スクリプトは自動的にシステムのバージョンを認識し、パッケージ管理を設定します。
    \item インストール時のパラメータ設定は不可ですが、本チュートリアルでは特に問題になりません。
    \item スクリプトはDockerの最新バージョンをインストールしますが、Dockerのアップグレードには使用できません。
\end{itemize}

\subsection{\texttt{apt}を使用したインストール}
\begin{enumerate}
    \item Dockerのリポジトリを設定
\begin{lstlisting}[language=bash]
# Add Docker's official GPG key:
sudo apt-get update
sudo apt-get install ca-certificates curl
sudo install -m 0755 -d /etc/apt/keyrings
sudo curl -fsSL https://download.docker.com/linux/ubuntu/gpg -o /etc/apt/keyrings/docker.asc
sudo chmod a+r /etc/apt/keyrings/docker.asc

# Add the repository to Apt sources:
echo \
  "deb [arch=$(dpkg --print-architecture) signed-by=/etc/apt/keyrings/docker.asc] https://download.docker.com/linux/ubuntu \
  (. /etc/os-release && echo \"(. /etc/os-release && echo \"{UBUNTU_CODENAME:-$VERSION_CODENAME}\") stable" | \
  sudo tee /etc/apt/sources.list.d/docker.list > /dev/null
sudo apt-get update
\end{lstlisting}

    \item 最新バージョンのDockerをインストール
\begin{lstlisting}[language=bash]
sudo apt-get install docker-ce docker-ce-cli containerd.io docker-buildx-plugin docker-compose-plugin
\end{lstlisting}
\end{enumerate}

アップデート時は、手順2を繰り返すだけでOKです。

\subsection{インストール確認}
\begin{lstlisting}[language=bash]
sudo docker run hello-world
\end{lstlisting}

出力結果は以下のようになります:
\begin{lstlisting}
Hello from Docker!
This message shows that your installation appears to be working correctly.
\end{lstlisting}

\subsection{ユーザーのDocker実行権限を設定}
上記の手順でインストールされたDockerは、デフォルトでは\texttt{root}ユーザーとして実行されます。そのため、\texttt{sudo}を毎回使用する必要があります。

Dockerをより便利に使用するために、現在のユーザーを\texttt{docker}グループに追加します。このグループはDockerのインストール時に自動作成されます。

\begin{enumerate}
    \item \texttt{docker}グループの確認と作成
\begin{lstlisting}[language=bash]
getent group docker
\end{lstlisting}

もしグループが存在しない場合、以下のコマンドで作成できます:
\begin{lstlisting}[language=bash]
sudo groupadd docker
\end{lstlisting}

    \item ユーザーを\texttt{docker}グループに追加
\begin{lstlisting}[language=bash]
sudo usermod -aG docker your_username
newgrp docker
\end{lstlisting}

\end{enumerate}

これにより、今後\texttt{sudo}を使わずにDockerコマンドを実行できるようになります。